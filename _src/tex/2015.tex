\nwfilename{_src/2015/day/25.nw}\nwbegindocs{0}\newpage% ===> this file was generated automatically by noweave --- better not edit it
\section{Day 25: Let It Snow}
\marginnote{\url{https://adventofcode.com/2015/day/25}}

Merry Christmas! Santa is booting up his weather machine; looks like you might
get a \href{https://adventofcode.com/2015/day/1}{white Christmas} after all.

The weather machine beeps! On the console of the machine is a copy protection
message asking you to
\href{https://en.wikipedia.org/wiki/Copy_protection\#Early_video_games}{enter a
  code from the instruction manual}. Apparently, it refuses to run unless you
give it that code. No problem; you'll just look up the code in the--

``Ho ho ho'', Santa ponders aloud. ``I can't seem to find the manual.''

You look up the support number for the manufacturer and give them a call. Good
thing, too - that 49th star wasn't going to earn itself.

``Oh, that machine is quite old!'', they tell you. ``That model went out of
support six minutes ago, and we just finished shredding all of the manuals. I
bet we can find you the code generation algorithm, though.''

After putting you on hold for twenty minutes (your call is \textbf{very}
important to them, it reminded you repeatedly), they finally find an engineer
that remembers how the code system works.

The codes are printed on an infinite sheet of paper\marginnote{The paper is very
  thin so it can be folded up neatly into the manual.}, starting in the top-left
corner. The codes are filled in by diagonals: starting with the first row with
an empty first box, the codes are filled in diagonally up and to the right. This
process repeats until the
\href{https://en.wikipedia.org/wiki/Cantor's_diagonal_argument}{infinite paper
  is covered}. So, the first few codes are filled in in this order:

\begin{table}[h]
  \centering
  \begin{tabular}{r|rrrrrr}
    & 1 & 2 & 3 & 4 & 5 & 6 \\
    \midrule
    1 & 1 & 3 & 6 & 10 & 15 & 21 \\
    2 & 2 & 5 & 9 & 14 & 20 \\
    3 & 4 & 8 & 13 & 19 \\
    4 & 7 & 12 & 18 \\
    5 & 11 & 17 \\
    6 & 16
  \end{tabular}
\end{table}

For example, the 12th code would be written to row \hs{4}, column \hs{2}; the
15th code would be written to row \hs{1}, column \hs{5}.

The voice on the other end of the phone continues with how the codes are
actually generated. The first code is \hs{20151125}. After that, each code is
generated by taking the previous one, multiplying it by \hs{252533}, and then
keeping the remainder from dividing that value by \hs{33554393}.

So, to find the second code (which ends up in row \hs{2}, column \hs{1}), start
with the previous value, \hs{20151125}. Multiply it by \hs{252533} to get
\hs{5088824049625}. Then, divide that by \hs{33554393}, which leaves a remainder
of \hs{31916031}. That remainder is the second code.

``Oh!'', says the voice. ``It looks like we missed a scrap from one of the
manuals. Let me read it to you.'' You write down his numbers:

\begin{table}[h]
  \centering
  \begin{tabular}{r|rrrrrr}
    & 1 & 2 & 3 & 4 & 5 & 6 \\
    \midrule
    1 & 20151125 & 18749137 & 17289845 & 30943339 & 10071777 & 33511524 \\
    2 & 31916031 & 21629792 & 16929656 & 7726640 & 15514188 & 4041754 \\
    3 &16080970 & 8057251 & 1601130 & 7981243 & 11661866 & 16474243 \\
    4 & 24592653 & 32451966 & 21345942 & 9380097 & 10600672 & 31527494 \\
    5 & 77061 & 17552253 & 28094349 & 6899651 & 9250759 & 31663883 \\
    6 & 33071741 & 6796745 & 25397450 & 24659492 & 1534922 & 27995004
  \end{tabular}
\end{table}

\newpage

``Now remember'', the voice continues, ``that's not even all of the first few
numbers; for example, you're missing the one at 7,1 that would come before
6,2. But, it should be enough to let your-- oh, it's time for lunch! Bye!'' The
call disconnects.

Santa looks nervous. Your puzzle input contains the message on the machine's
console. \textbf{What code do you give the machine?}

\newthought{Part Two}

The machine springs to life, then falls silent again. It beeps. ``Insufficient
fuel'', the console reads. ``\textbf{Fifty stars} are required before
proceeding. \textbf{One star} is available.''

...``one star is available''? You check the fuel tank; sure enough, a lone star
sits at the bottom, awaiting its friends. Looks like you need to provide 49
yourself.
\nwenddocs{}\nwfilename{_src/2015/haskell/25.nw}\nwbegindocs{0}\subsection{Haskell solution}

Rather than try to generate the codes sequentially, which can quickly result in
a stack overflow, make some observations.

\newthought{The positions in first column} are the
\hrefootnote{https://en.wikipedia.org/wiki/Lazy_caterer's_sequence}{lazy
  caterer's sequence}, i.e. \hrefootnote{https://oeis.org/A000124}{A000124 in
  The Online Encyclopedia of Integer Sequences}.

\begin{equation}
  a(n) = n \times (n - 1) / 2 + 1
\end{equation}

Or equivalently in Haskell:

\nwenddocs{}\nwbegincode{1}\sublabel{NW22O72u-2mhKol-1}\nwmargintag{{\nwtagstyle{}\subpageref{NW22O72u-2mhKol-1}}}\moddef{Define a000124~{\nwtagstyle{}\subpageref{NW22O72u-2mhKol-1}}}\endmoddef\nwstartdeflinemarkup\nwusesondefline{\\{NW22O72u-1XqXnB-1}}\nwprevnextdefs{\relax}{NW22O72u-2mhKol-2}\nwenddeflinemarkup
\nwlinkedidentc{a000124}{NW22O72u-2mhKol-1} :: (Integral a) => a -> a
\nwlinkedidentc{a000124}{NW22O72u-2mhKol-1} n = n * (n - 1) `div` 2 + 1
\nwindexdefn{\nwixident{a000124}}{a000124}{NW22O72u-2mhKol-1}\eatline
\nwalsodefined{\\{NW22O72u-2mhKol-2}}\nwused{\\{NW22O72u-1XqXnB-1}}\nwidentdefs{\\{{\nwixident{a000124}}{a000124}}}\nwendcode{}\nwbegindocs{2}\nwdocspar
Since this problem might involve some rather large numbers, specialize the
polymorphic {\Tt{}\nwlinkedidentq{a000124}{NW22O72u-2mhKol-1}\nwendquote} to Haskell's arbitrary precision \hs{Integer}, and
ensure the compiler inlines it.

\nwenddocs{}\nwbegincode{3}\sublabel{NW22O72u-2mhKol-2}\nwmargintag{{\nwtagstyle{}\subpageref{NW22O72u-2mhKol-2}}}\moddef{Define a000124~{\nwtagstyle{}\subpageref{NW22O72u-2mhKol-1}}}\plusendmoddef\nwstartdeflinemarkup\nwusesondefline{\\{NW22O72u-1XqXnB-1}}\nwprevnextdefs{NW22O72u-2mhKol-1}{\relax}\nwenddeflinemarkup
\{-# SPECIALIZE INLINE \nwlinkedidentc{a000124}{NW22O72u-2mhKol-1} :: Integer -> Integer #-\}
\nwused{\\{NW22O72u-1XqXnB-1}}\nwidentuses{\\{{\nwixident{a000124}}{a000124}}}\nwindexuse{\nwixident{a000124}}{a000124}{NW22O72u-2mhKol-2}\nwendcode{}\nwbegindocs{4}\nwdocspar

\newthought{Then to calculate the distance} from that position to that of
another column $c$ in the same row $r$, simply subtract the $r$th
\hrefootnote{https://en.wikipedia.org/wiki/Triangular_number}{triangular number}
from the $(c + r - 1)$th. The triangular numbers are known as
\hrefootnote{https://oeis.org/A000217}{A000217 in The Online Encyclopedia of
  Integer Sequences}.

\begin{equation}
  T(n) = n \times (n + 1) / 2
\end{equation}

Or equivalently in Haskell:

\nwenddocs{}\nwbegincode{5}\sublabel{NW22O72u-QCx5X-1}\nwmargintag{{\nwtagstyle{}\subpageref{NW22O72u-QCx5X-1}}}\moddef{Define a000217~{\nwtagstyle{}\subpageref{NW22O72u-QCx5X-1}}}\endmoddef\nwstartdeflinemarkup\nwusesondefline{\\{NW22O72u-1XqXnB-1}}\nwprevnextdefs{\relax}{NW22O72u-QCx5X-2}\nwenddeflinemarkup
\nwlinkedidentc{a000217}{NW22O72u-QCx5X-1} :: (Integral a) => a -> a
\nwlinkedidentc{a000217}{NW22O72u-QCx5X-1} n = n * (n + 1) `div` 2
\nwindexdefn{\nwixident{a000217}}{a000217}{NW22O72u-QCx5X-1}\eatline
\nwalsodefined{\\{NW22O72u-QCx5X-2}}\nwused{\\{NW22O72u-1XqXnB-1}}\nwidentdefs{\\{{\nwixident{a000217}}{a000217}}}\nwendcode{}\nwbegindocs{6}\nwdocspar
Just as before, specialize to \hs{Integer} and inline.

\nwenddocs{}\nwbegincode{7}\sublabel{NW22O72u-QCx5X-2}\nwmargintag{{\nwtagstyle{}\subpageref{NW22O72u-QCx5X-2}}}\moddef{Define a000217~{\nwtagstyle{}\subpageref{NW22O72u-QCx5X-1}}}\plusendmoddef\nwstartdeflinemarkup\nwusesondefline{\\{NW22O72u-1XqXnB-1}}\nwprevnextdefs{NW22O72u-QCx5X-1}{\relax}\nwenddeflinemarkup
\{-# SPECIALIZE INLINE \nwlinkedidentc{a000217}{NW22O72u-QCx5X-1} :: Integer -> Integer #-\}
\nwused{\\{NW22O72u-1XqXnB-1}}\nwidentuses{\\{{\nwixident{a000217}}{a000217}}}\nwindexuse{\nwixident{a000217}}{a000217}{NW22O72u-QCx5X-2}\nwendcode{}\nwbegindocs{8}\nwdocspar

So, to find the position in the sequence of codes from a given column and row:

\nwenddocs{}\nwbegincode{9}\sublabel{NW22O72u-6ilA0-1}\nwmargintag{{\nwtagstyle{}\subpageref{NW22O72u-6ilA0-1}}}\moddef{Find the position~{\nwtagstyle{}\subpageref{NW22O72u-6ilA0-1}}}\endmoddef\nwstartdeflinemarkup\nwusesondefline{\\{NW22O72u-2c4Gzh-1}}\nwenddeflinemarkup
position = \nwlinkedidentc{a000124}{NW22O72u-2mhKol-1} row + \nwlinkedidentc{a000217}{NW22O72u-QCx5X-1} (column + row - 1) - \nwlinkedidentc{a000217}{NW22O72u-QCx5X-1} row
\nwused{\\{NW22O72u-2c4Gzh-1}}\nwidentuses{\\{{\nwixident{a000124}}{a000124}}\\{{\nwixident{a000217}}{a000217}}}\nwindexuse{\nwixident{a000124}}{a000124}{NW22O72u-6ilA0-1}\nwindexuse{\nwixident{a000217}}{a000217}{NW22O72u-6ilA0-1}\nwendcode{}\nwbegindocs{10}\nwdocspar

\newthought{The generating function} for the sequence is as follows, where $n
\in \mathbb{N}$ is the position.

\begin{equation}
  f(n) =
  \begin{cases}
    20151125, & \text{for } n=0 \\
    f(n - 1) \times 252533 \pmod{33554393}, & \text{otherwise}
  \end{cases}
\end{equation}

That recursive definition can be rewritten as a closed formula.

\begin{equation}
  f(n) = 20151125 \times 252533^{n-1} \pmod{33554393}
\end{equation}

Or equivalently in Haskell:

\nwenddocs{}\nwbegincode{11}\sublabel{NW22O72u-3sqonN-1}\nwmargintag{{\nwtagstyle{}\subpageref{NW22O72u-3sqonN-1}}}\moddef{Find the code at a given position~{\nwtagstyle{}\subpageref{NW22O72u-3sqonN-1}}}\endmoddef\nwstartdeflinemarkup\nwusesondefline{\\{NW22O72u-2c4Gzh-1}}\nwenddeflinemarkup
20151125 * (252533 ^ (position - 1)) `mod` 33554393
\nwused{\\{NW22O72u-2c4Gzh-1}}\nwendcode{}\nwbegindocs{12}\nwdocspar

Thus, Part One can be solved by implementing a function that takes a coordinate
pair and returns the specified code.

\nwenddocs{}\nwbegincode{13}\sublabel{NW22O72u-2c4Gzh-1}\nwmargintag{{\nwtagstyle{}\subpageref{NW22O72u-2c4Gzh-1}}}\moddef{Define partOne~{\nwtagstyle{}\subpageref{NW22O72u-2c4Gzh-1}}}\endmoddef\nwstartdeflinemarkup\nwusesondefline{\\{NW22O72u-1XqXnB-1}}\nwenddeflinemarkup
\nwlinkedidentc{partOne}{NW22O72u-2c4Gzh-1} :: \nwlinkedidentc{Coordinates}{NW22O72u-14h5MM-1} -> Integer
\nwlinkedidentc{partOne}{NW22O72u-2c4Gzh-1} (column, row) = \LA{}Find the code at a given position~{\nwtagstyle{}\subpageref{NW22O72u-3sqonN-1}}\RA{}
  where
    \LA{}Find the position~{\nwtagstyle{}\subpageref{NW22O72u-6ilA0-1}}\RA{}
\nwindexdefn{\nwixident{partOne}}{partOne}{NW22O72u-2c4Gzh-1}\eatline
\nwused{\\{NW22O72u-1XqXnB-1}}\nwidentdefs{\\{{\nwixident{partOne}}{partOne}}}\nwidentuses{\\{{\nwixident{Coordinates}}{Coordinates}}}\nwindexuse{\nwixident{Coordinates}}{Coordinates}{NW22O72u-2c4Gzh-1}\nwendcode{}\nwbegindocs{14}\nwdocspar
There is nothing to solve for Part Two.

\newthought{To parse the input}, write a silly, overly explicit \hs{Parser}.

\nwenddocs{}\nwbegincode{15}\sublabel{NW22O72u-4Uy8SS-1}\nwmargintag{{\nwtagstyle{}\subpageref{NW22O72u-4Uy8SS-1}}}\moddef{Import tools for parsing the input~{\nwtagstyle{}\subpageref{NW22O72u-4Uy8SS-1}}}\endmoddef\nwstartdeflinemarkup\nwusesondefline{\\{NW22O72u-1XqXnB-1}}\nwenddeflinemarkup
import Control.Monad (void)
import Text.Trifecta (Parser, comma, natural, symbol)
\nwused{\\{NW22O72u-1XqXnB-1}}\nwendcode{}\nwbegindocs{16}\nwdocspar

\nwenddocs{}\nwbegincode{17}\sublabel{NW22O72u-3dpu5v-1}\nwmargintag{{\nwtagstyle{}\subpageref{NW22O72u-3dpu5v-1}}}\moddef{Define coordinates~{\nwtagstyle{}\subpageref{NW22O72u-3dpu5v-1}}}\endmoddef\nwstartdeflinemarkup\nwusesondefline{\\{NW22O72u-1XqXnB-1}}\nwenddeflinemarkup
\nwlinkedidentc{coordinates}{NW22O72u-3dpu5v-1} :: Parser \nwlinkedidentc{Coordinates}{NW22O72u-14h5MM-1}
\nwlinkedidentc{coordinates}{NW22O72u-3dpu5v-1} =
  do
    void $ symbol "To continue, please consult the code grid in the manual."
    row <- symbol "Enter the code at row" *> natural <* comma
    column <- symbol "column" *> natural
    pure (column, row)
\nwindexdefn{\nwixident{coordinates}}{coordinates}{NW22O72u-3dpu5v-1}\eatline
\nwused{\\{NW22O72u-1XqXnB-1}}\nwidentdefs{\\{{\nwixident{coordinates}}{coordinates}}}\nwidentuses{\\{{\nwixident{Coordinates}}{Coordinates}}}\nwindexuse{\nwixident{Coordinates}}{Coordinates}{NW22O72u-3dpu5v-1}\nwendcode{}\nwbegindocs{18}\nwdocspar
{\Tt{}\nwlinkedidentq{Coordinates}{NW22O72u-14h5MM-1}\nwendquote} is just a two-tuple of \hs{Integer}s.

\nwenddocs{}\nwbegincode{19}\sublabel{NW22O72u-14h5MM-1}\nwmargintag{{\nwtagstyle{}\subpageref{NW22O72u-14h5MM-1}}}\moddef{Define Coordinates~{\nwtagstyle{}\subpageref{NW22O72u-14h5MM-1}}}\endmoddef\nwstartdeflinemarkup\nwusesondefline{\\{NW22O72u-1XqXnB-1}}\nwenddeflinemarkup
type \nwlinkedidentc{Coordinates}{NW22O72u-14h5MM-1} = (Integer, Integer)
\nwindexdefn{\nwixident{Coordinates}}{Coordinates}{NW22O72u-14h5MM-1}\eatline
\nwused{\\{NW22O72u-1XqXnB-1}}\nwidentdefs{\\{{\nwixident{Coordinates}}{Coordinates}}}\nwendcode{}\nwbegindocs{20}\nwdocspar
Define the usual {\Tt{}\nwlinkedidentq{getInput}{NW22O72u-2tfAyb-1}\nwendquote}.

\nwenddocs{}\nwbegincode{21}\sublabel{NW22O72u-17Bmws-1}\nwmargintag{{\nwtagstyle{}\subpageref{NW22O72u-17Bmws-1}}}\moddef{Import some common utilities~{\nwtagstyle{}\subpageref{NW22O72u-17Bmws-1}}}\endmoddef\nwstartdeflinemarkup\nwusesondefline{\\{NW22O72u-1XqXnB-1}}\nwenddeflinemarkup
import AdventOfCode.Input (parseInput)
import AdventOfCode.TH (inputFilePath)
\nwused{\\{NW22O72u-1XqXnB-1}}\nwendcode{}\nwbegindocs{22}\nwdocspar

\nwenddocs{}\nwbegincode{23}\sublabel{NW22O72u-2tfAyb-1}\nwmargintag{{\nwtagstyle{}\subpageref{NW22O72u-2tfAyb-1}}}\moddef{Define getInput~{\nwtagstyle{}\subpageref{NW22O72u-2tfAyb-1}}}\endmoddef\nwstartdeflinemarkup\nwusesondefline{\\{NW22O72u-1XqXnB-1}}\nwenddeflinemarkup
\nwlinkedidentc{getInput}{NW22O72u-2tfAyb-1} :: IO \nwlinkedidentc{Coordinates}{NW22O72u-14h5MM-1}
\nwlinkedidentc{getInput}{NW22O72u-2tfAyb-1} = parseInput \nwlinkedidentc{coordinates}{NW22O72u-3dpu5v-1} $(inputFilePath)
\nwindexdefn{\nwixident{getInput}}{getInput}{NW22O72u-2tfAyb-1}\eatline
\nwused{\\{NW22O72u-1XqXnB-1}}\nwidentdefs{\\{{\nwixident{getInput}}{getInput}}}\nwidentuses{\\{{\nwixident{Coordinates}}{Coordinates}}\\{{\nwixident{coordinates}}{coordinates}}}\nwindexuse{\nwixident{Coordinates}}{Coordinates}{NW22O72u-2tfAyb-1}\nwindexuse{\nwixident{coordinates}}{coordinates}{NW22O72u-2tfAyb-1}\nwendcode{}\nwbegindocs{24}\nwdocspar
\newthought{Bring it} all together.

\nwenddocs{}\nwbegincode{25}\sublabel{NW22O72u-1XqXnB-1}\nwmargintag{{\nwtagstyle{}\subpageref{NW22O72u-1XqXnB-1}}}\moddef{Day25.hs~{\nwtagstyle{}\subpageref{NW22O72u-1XqXnB-1}}}\endmoddef\nwstartdeflinemarkup\nwenddeflinemarkup
module AdventOfCode.Year2015.Day25 where

\LA{}Import some common utilities~{\nwtagstyle{}\subpageref{NW22O72u-17Bmws-1}}\RA{}
\LA{}Import tools for parsing the input~{\nwtagstyle{}\subpageref{NW22O72u-4Uy8SS-1}}\RA{}

\LA{}Define Coordinates~{\nwtagstyle{}\subpageref{NW22O72u-14h5MM-1}}\RA{}

main :: IO ()
main =
  do
    putStr "Part One: "
    print . \nwlinkedidentc{partOne}{NW22O72u-2c4Gzh-1} =<< \nwlinkedidentc{getInput}{NW22O72u-2tfAyb-1}

\LA{}Define partOne~{\nwtagstyle{}\subpageref{NW22O72u-2c4Gzh-1}}\RA{}

\LA{}Define getInput~{\nwtagstyle{}\subpageref{NW22O72u-2tfAyb-1}}\RA{}

\LA{}Define coordinates~{\nwtagstyle{}\subpageref{NW22O72u-3dpu5v-1}}\RA{}

\LA{}Define a000124~{\nwtagstyle{}\subpageref{NW22O72u-2mhKol-1}}\RA{}

\LA{}Define a000217~{\nwtagstyle{}\subpageref{NW22O72u-QCx5X-1}}\RA{}
\nwnotused{Day25.hs}\nwidentuses{\\{{\nwixident{getInput}}{getInput}}\\{{\nwixident{partOne}}{partOne}}}\nwindexuse{\nwixident{getInput}}{getInput}{NW22O72u-1XqXnB-1}\nwindexuse{\nwixident{partOne}}{partOne}{NW22O72u-1XqXnB-1}\nwendcode{}\nwbegindocs{26}\nwdocspar
\nwenddocs{}

\nwixlogsorted{c}{{Day25.hs}{NW22O72u-1XqXnB-1}{\nwixd{NW22O72u-1XqXnB-1}}}%
\nwixlogsorted{c}{{Define a000124}{NW22O72u-2mhKol-1}{\nwixd{NW22O72u-2mhKol-1}\nwixd{NW22O72u-2mhKol-2}\nwixu{NW22O72u-1XqXnB-1}}}%
\nwixlogsorted{c}{{Define a000217}{NW22O72u-QCx5X-1}{\nwixd{NW22O72u-QCx5X-1}\nwixd{NW22O72u-QCx5X-2}\nwixu{NW22O72u-1XqXnB-1}}}%
\nwixlogsorted{c}{{Define Coordinates}{NW22O72u-14h5MM-1}{\nwixd{NW22O72u-14h5MM-1}\nwixu{NW22O72u-1XqXnB-1}}}%
\nwixlogsorted{c}{{Define coordinates}{NW22O72u-3dpu5v-1}{\nwixd{NW22O72u-3dpu5v-1}\nwixu{NW22O72u-1XqXnB-1}}}%
\nwixlogsorted{c}{{Define getInput}{NW22O72u-2tfAyb-1}{\nwixd{NW22O72u-2tfAyb-1}\nwixu{NW22O72u-1XqXnB-1}}}%
\nwixlogsorted{c}{{Define partOne}{NW22O72u-2c4Gzh-1}{\nwixd{NW22O72u-2c4Gzh-1}\nwixu{NW22O72u-1XqXnB-1}}}%
\nwixlogsorted{c}{{Find the code at a given position}{NW22O72u-3sqonN-1}{\nwixd{NW22O72u-3sqonN-1}\nwixu{NW22O72u-2c4Gzh-1}}}%
\nwixlogsorted{c}{{Find the position}{NW22O72u-6ilA0-1}{\nwixd{NW22O72u-6ilA0-1}\nwixu{NW22O72u-2c4Gzh-1}}}%
\nwixlogsorted{c}{{Import some common utilities}{NW22O72u-17Bmws-1}{\nwixd{NW22O72u-17Bmws-1}\nwixu{NW22O72u-1XqXnB-1}}}%
\nwixlogsorted{c}{{Import tools for parsing the input}{NW22O72u-4Uy8SS-1}{\nwixd{NW22O72u-4Uy8SS-1}\nwixu{NW22O72u-1XqXnB-1}}}%
\nwixlogsorted{i}{{\nwixident{a000124}}{a000124}}%
\nwixlogsorted{i}{{\nwixident{a000217}}{a000217}}%
\nwixlogsorted{i}{{\nwixident{Coordinates}}{Coordinates}}%
\nwixlogsorted{i}{{\nwixident{coordinates}}{coordinates}}%
\nwixlogsorted{i}{{\nwixident{getInput}}{getInput}}%
\nwixlogsorted{i}{{\nwixident{partOne}}{partOne}}%

